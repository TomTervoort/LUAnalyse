\documentclass[a4paper]{article}

\usepackage[english]{babel}
\usepackage[a4paper]{geometry}
\usepackage{amsmath}
\usepackage{graphicx}
\usepackage{comment}
\usepackage{hyperref}
\usepackage{url}

\title{Documentation Dataflow And Abstract Interpretation, Automatic Program Analysis}
\author{Stijn van Drongelen (3117537), Tom Tervoort (), Gerben van Veenendaal (3460692)}
\date{\today}

\begin{document}

\maketitle

\section*{Introduction}

For this assignment, we chose to implement \textit{soft typing} for the
programming language Lua. Lua is very popular as a simple scripting language,
for example to implement plugins for various programs. We chose Lua because it
is a very elegant language, simple, yet powerful.

The implementation of this analysis was done in a combination of Lua and
Haskell. The Lua portion is only responsible for parsing Lua code; we borrowed
the code from an existing Lua parser by Mark Langen%
\footnote{\url{https://github.com/stravant/LuaMinify}}.

\section*{Lua's type system}

Ignoring advanced concepts, Lua has only six types:

\begin{itemize}
\item \texttt{nil} is used to signify nonexistence. For instance, any
    non-existing variable has the value \texttt{nil}. One peculiarity that
    illustrates the ``nonexistence'' semantics is that trying to insert
    a value with index \texttt{nil} into a table leads to an error.
    We say that \texttt{nil} is of type \textit{nil}.
\item The \textit{boolean} type is inhabited by \texttt{true} and \texttt{false}.
\item The \textit{number} type is the set of all possible double-precision
    floating point numbers.
\item The \textit{string} type is the set of all sequences of ``characters'',
    where one character is at least eight bits.%
    \footnote{More information about Lua strings: \url{http://lua-users.org/wiki/LuaUnicode}}
\item The \textit{table} type is a mutable mapping from any non-\texttt{nil}
    value to any non-\texttt{nil} value.\footnote{Any key value that is not
    explicitly in the table, including \texttt{nil}, maps to \texttt{nil}.}
\item Lua functions are first-class citizens, so we also need to consider them
    as inhabitants of the \textit{function} type.
\end{itemize}

Although Lua's semantics are lenient when it comes to its types, some operations
can lead to type errors. For example, arithmetic operators only work with numbers,
table operations fail with anything besides tables, and only functions are callable.
These properties can be used to approximate which types the variables in
the program may have if the program won't crash on a type error.

\section*{Analysis}

In our analysis, we keep track of the possible types a variable may have,
with the assumption that the program will not result in a type error.
The type lattice we use is product the lattices of the six Lua types, using the
product order ($(a,b) \leq (a',b') \leftrightarrow a \leq a' \wedge b \leq b'$).
The precision of these lattices varies greatly:

\begin{itemize}
\item Because the \textit{nil} and \textit{boolean} types have very few
    inhabitants, we take the powerset of these as their type lattice,
    which results in exact precision.
\item We initially planned to keep track of certain properties of
    \textit{number}, but we never implemented it, which leaves us with only
    bottom (may be any number) and top (may be no number).
\item For \textit{string}, we consider small sets of constants to be between
    top (all possible strings) and bottom (not a string at all). If the set
    becomes too large after a join, we replace the result with top.

    Implementing this allowed for a simplification in treating tables.
    Even when only single string constants are allowed in the lattice,
    we can treat the notation \texttt{table.member} as \texttt{table["member"]}
    without any loss of precision.
\item The lattice for our \textit{table} type is the most intricate in our
    system. Besides the bottom (not a table) and top (everything may map to
    anything), we keep track of the values behind (known) constant indices and
    (unknown) variable indices.
\item Although our \textit{function} type lattice is relatively complex as it
    is, keeping track of admissable types of parameters and return values, and
    effects on global variables, we don't actually use it. Effectively,
    we only use the bottom and top.
\end{itemize}

\section*{Strong points}

One strong point of our implementation is that we support a big subset of the language.


\section*{Weak points}

\section*{Running the examples}



\end{document}



\begin{comment}

The programming assignments should be in the form of a single zip file including the necessary code and documentation. Every practical assignment should include not only the code, but also a high level description of that code (so that I can quickly learn my way round), a suitable set of example programs (and sometimes example analyses). Try to be thorough in explaining why your solution works and to which extent it does. Be clear about your limitations and your strong points. If you have done anything special that you think qualifies for a bonus, tell me about it explicitly. Also include a description how I may build your software, and which packages I need to get, where they come from and how I should compile them. If you prefer, because you think it may very troublesome to get everything up and running on my MacBook? , you can also choose to demonstrate your implementation on your own machine.



In the end, I want not only the code, but also: a range of examples, and documentation containing a talk through of one of the examples, the restrictions you imposed, what your lattice is, how join is defined, whether you use widening and what the operator is, the architecture of the implementation (what can be found where in your code), how to compile, how to run, what packages and tools do I need, major design choices (flow sensitive or not, context sensitive or not), etc.

\end{comment}
