\documentclass[a4paper]{article}

\usepackage[english]{babel}
\usepackage[a4paper]{geometry}
\usepackage{amsmath}
\usepackage{graphicx}
\usepackage{comment}
\usepackage{hyperref}
\usepackage{url}

\title{Documentation Dataflow And Abstract Interpretation, Automatic Program Analysis}
\author{Stijn van Drongelen (), Tom Tervoort (), Gerben van Veenendaal (3460692)}
\date{\today}

\begin{document}

\maketitle

\section*{Analysis language}

For the assignment, we chose to implement \textit{soft typing} for the programming language LUA. LUA is very popular as a simple scripting language, for example to implement plugins for various programs. We chose LUA because it is a very elegant language, simple, yet powerful.

\section*{Implementation language}

The implementation language is both LUA and Haskell. We have found a LUA parser at \url{https://github.com/stravant/LuaMinify}

\section*{Analysis}




\section*{Strong points}

One strong point of our implementation is that we support a big subset of the language.


\section*{Weak points}

\section*{Running the examples}



\end{document}



\begin{comment}

The programming assignments should be in the form of a single zip file including the necessary code and documentation. Every practical assignment should include not only the code, but also a high level description of that code (so that I can quickly learn my way round), a suitable set of example programs (and sometimes example analyses). Try to be thorough in explaining why your solution works and to which extent it does. Be clear about your limitations and your strong points. If you have done anything special that you think qualifies for a bonus, tell me about it explicitly. Also include a description how I may build your software, and which packages I need to get, where they come from and how I should compile them. If you prefer, because you think it may very troublesome to get everything up and running on my MacBook? , you can also choose to demonstrate your implementation on your own machine.



In the end, I want not only the code, but also: a range of examples, and documentation containing a talk through of one of the examples, the restrictions you imposed, what your lattice is, how join is defined, whether you use widening and what the operator is, the architecture of the implementation (what can be found where in your code), how to compile, how to run, what packages and tools do I need, major design choices (flow sensitive or not, context sensitive or not), etc.

\end{comment}